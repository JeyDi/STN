\section*{Introduzione}
\addcontentsline{toc}{section}{Introduzione}

    Oggigiorno i social network rivestono un ruolo chiave nella vita di tutti i giorni, essi infatti influenzano tutti gli aspetti della nostra vita: dal marketing allo shopping, passando per le interazioni con le celebrità ed aziende fino a sostituire, in molte occasioni, le fonti di informazioni primarie come quotidiani o notiziari.
    
    Un report del 2018 effettuato dal Pew Internet research center \cite{ReportPewInternetResearch2018} ha verificato che della popolazione italiana il 64\% usa i social media per informarsi su fatti di cronaca e l’81\% lo usa per informarsi e ricercare pareri su servizi o brand.
    
    Tuttavia, soprattutto negli ultimi anni, i social hanno rivestito un ruolo chiave anche nel mondo politico, sia a livello di opinioni politiche che di possibilità di influenza sul voto degli elettori; sono presenti numerosi studi che hanno indagato tali fenomeni.
    
    "Activism in the social media age" \cite{ActivismSocialMedia} ha preso in esame il 53\% della popolazione americana che nell'arco di un anno ha utilizzato i social per ragioni politiche o sociali. Inoltre, da altri sondaggi è emerso che il 69\% della popolazione pensa che i social siano importanti per informare i politici dei problemi che i cittadini si trovano ad affrontare; e il 58\% ritiene che i social media influenzano le decisioni politiche.
    
    Su quest'ultimo punto un’altra importante ricerca "Automating power: Social bot interference in global politics" \cite{AutomaticPower} scende più nello specifico indagando come i Bot sono stati importanti nel veicolare pensieri politici e quindi quanto siano potenzialmente necessari e utili per influenzare le masse. Tale ricerca prende in esame le elezioni di vari paesi e,sulla base di altre ricerche effettuate nel corso degli anni, definisce a quale scopo sono stati impiegati i political bot e da chi probabilmente sono stati attivati (se dallo stato stesso o da aziende esterne).
    
    Lo studio effettuato, le cui specifiche sono descritte nel presente documento, ha l’obiettivo di quantificare l’importanza di selezionare, all’interno di una determinata rete, gli utenti che sono nella posizione migliore di influenzare le masse e che quindi trasformandoli in bot permettono di massimizzare la propagazione di una determinata notizia o pensiero.