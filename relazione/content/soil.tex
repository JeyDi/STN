\section{Soil}
    Soil è un modulo Python ABSS\footnote{Agent-Based Social Simulation}. Ciò vuol dire che permette di sfruttare uno scheletro di base per lanciare simulazioni di modelli sociali tramite la personalizzazione degli agenti.
    
    È stato scelto Soil in quanto è un progetto open source, costantemente in crescita e aggiornato, cross-platform e permette di utilizzare standard di moduli già rodati e ampiamente utilizzati e conosciuti.
    
    Secondo le nostre ricerche, la documentazione di Soil non risulta sempre corretta e aggiornata, tuttavia essendo un progetto open source è possibile ottenere informazioni relative al funzionamento di determinate utility in modo abbastanza agevole.
    Oltre alla documentazione, Soil offre una serie di esempi di personalizzazione e caratterizzazione degli agenti, con i corrispettivi file YAML che permettono di specificare una serie di parametri necessari per poter definire le varie casistiche della simulazione.
    
    Nel capitolo seguente verrà spiegato il funzionamento di Soil per quanto riguarda l’uso che ne è stato fatto durante lo sviluppo di questo progetto. Adottando l'utilizzo dei file di configurazione possiamo definire quindi due componenti principali nello sviluppo di una simulazione:
    
    \begin{itemize}
        \item un modulo python per la definizione degli agenti veri e propri, con i relativi stati e varie funzionalità per definirne il comportamento;
        \item un file di configurazione YAML (per ogni simulazione) dedito alla definizioni di specifici parametri.
    \end{itemize}
    
    % aggiungere informazioni dalla documentazione di Soil (immagine schema, nozioni da "Configuring a simulation" e "Tutorial") https://soilsim.readthedocs.io/en/latest/index.html
