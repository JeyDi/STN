\section{Conclusioni}
    La maggior diffusione della news all’interno del grafo si è ottenuta selezionando i Bot secondo la misura di centralità della Betweenness, è stata anche l’unica configurazione nella quale gli utenti infetti o esposti hanno superato il numero di utenti non esposti.\newline 
    In tutte le simulazioni l’infezione è avvenuta per la maggior parte in maniera diretta (cioè dalla propagazione della notizia partita dall’opinion leader o da un bot).
    \newline
    Per quanto riguarda la fase di validazione non è stata possibile attuarla. A causa delle forti limitazioni nell'uso di Twitter dovute ad un restringimento delle policy di sicurezza e di accesso non siamo riusciti a validare il modello su casi reali. Le informazioni utilizzate all'interno delle analisi e la topologia della rete sono state ottenute effettuando uno scraping limitato proprio sui dati di twitter. Per la validazione del modello occorrerebbe avere a disposizione più dati e informazioni in modo da migliorare la fase di validazione e valutazione rispetto, ad esempio, a delle situazioni realistiche di diffusione e contaminazione di notizie che si sono verificate nel tempo.