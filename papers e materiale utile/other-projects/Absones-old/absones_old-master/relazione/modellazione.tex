\newcommand{\activity}{\cukeyword{activity}}
\newcommand{\topic}{\cukeyword{topic}}
\newcommand{\PI}{\cukeyword{PI}}
\newcommand{\TZ}{\cukeyword{TZ}}
\newcommand{\followers}{\cukeyword{followers}}
\newcommand{\following}{\cukeyword{following}}
\newcommand{\FOV}{\cukeyword{FOV}}
\newcommand{\dtag}{\cukeyword{dtag}}
\newcommand{\attach}{\cukeyword{attachment}}
\newcommand{\twt}{\cukeyword{tweet}}
\newcommand{\retwt}{\cukeyword{retweet}}
\newcommand{\lika}{\cukeyword{likability}}
\newcommand{\disla}{\cukeyword{dislikability}}
\newcommand{\interest}{\cukeyword{interest}}

\section{Modellazione}
\label{sec:model}
Per la creazione del modello deriviamo delle costanti che saranno
sempre vere:

\begin{description}
  \item[Fasi Temporali (FT)] Dividiamo la giornata in 12 fasi che 
    rappresentano 2 ore l'una in cui ogni utente ha un personale
    valore di \activity{} che rappresenta il suo utilizzo di Twitter.
    Tali fasi sono cos\`i suddivise:
    \begin{itemize}
      \item 4 fasi con \activity{} pari a 0 corrispondente a 8
      ore di sonno, ovvero la quantit\`a consigliata (e quasi mai 
      rispettata).
      \item 4 fasi con \activity{} bassa, corrispondente a 8 ore di 
      lavoro
      \item 4 fasi con \activity{} elevata che rappresenta 8 ore di 
      tempo libero in cui l'utente ha una attivit\`a maggiore su 
      Twitter
    \end{itemize} 
    Al momento ignoramo il weekend, ma valuteremo una possibilit\`a
     di definirlo e 
     implementarlo nelle prossime versioni.
  \item[Topic] ovvero i possibili argomenti di interesse esistenti 
    nel modello perci\`o abbiamo una lista \topic{} 
    $ = (t_1, \dots , t_n) $ definiti a priori e immutabili. 
    Non vengono mai utilizzati direttamente, ma vengono utilizzati 
    come indici di riferimento dunque non \`e necessario 
    implementarli realmente.
\end{description}



\subsection{Utenti}
\label{subsec:users}
Gli utenti sono rappresentati dai nodi della rete. Vediamo ora dunque
in che modo viene generato ogni nodo:

\begin{description}
  \item[Personal Interest (PI)] \`e una lista \PI{} 
    $=(p_1, \dots, p_n)$ dove $p_i \in (0, 1)$ 
    indica la probabilit\`a di interesse del nodo rispetto al 
    \topic{} $i$ generata casualmente. Al momento tale valore \`e immutabile.
  \item[Timezone (TZ)] ovvero una lista \TZ{} $=(tz_1, \dots, tz_{12}) $
  dove i $tz_i$ sono generati in accordo con le \textbf{FT} generati 
  come segue:
  \begin{itemize}
    \item Genero $1\leq i \leq 12$ casualmente
    \item Imposto le ore di sonno $tz_j = 0$ per $j = i, \dots , i+3$
    \item Imposto le ore di lavoro $tz_j$ = bassa \activity{} per 
      $j = i+4, \dots , i+7$
    \item Imposto le ore di tempo libero $tz_j$ = alta \activity{}
      per $j = i+8, \dots , i+11$
  \end{itemize}
  Tutte le precedenti operazioni di indici sono da considerarsi 
  $mod \quad 12$. Il motivo di questa scelta \`e quello di simulare
  sia diversi orari per le persone, sia diversi fusi orari.
\end{description}

Per poter fare operazioni sugli utenti abbiamo inoltre bisogno delle 
seguenti definizioni:

\begin{description}
  \item[Followers] per ogni utente U definiamo come \followers{}(U) 
  l'insieme dei followers di U.
  \item[Following] similmente \following{}(U) indica
  l'insieme degli utenti seguiti da U.
  \item[Direct Tag (dtag)] \dtag{}(U, T) \`e l'insieme dei tweets che
  contengo un tag all'utente U generati al tempo T. 
  Sar\'a spiegato meglio nella sezione
  riguardante i tweet [\ref{subsec:social}].
  \item[Interessi] per ogni utente U \interest{}(U) \`e l'insieme
  dei \topic{} che interessano a U.
  \item[Field Of View (FOV)] \FOV{}(U, T) rappresenta l'insieme dei
  delle notizie visualizzate dall'utente U al tempo T, con 
  \FOV{} $\subseteq$ \twt{}(\following{}(U), $T_i$) $\cup$ 
  \retwt{}(\following{}(U), $T_j$) $\cup$ \dtag{}(U, $T_l$) 
  per $T_{i,j,l} \in \overline{T}$ lista di
  tempi non maggiori al tempo attuale T.
  Questo insieme \`e definito sia per una questione
  computazionale, ma anche per un motivo reale in quanto \`e 
  difficile che un utente nell'arco della giornata riesca a vedere
  tutti i tweets e i retweets degli utenti che segue e i tweet che
  in cui risulta direttamente \textit{taggato}. Questa affermazione
  diventa sempre pi\`u ragionevole al crescere della popolarit\`a 
  di un utente.
  
\end{description}

Le relazioni tra gli utenti sono descritti dagli archi tra di essi.
\begin{description}
  \item[Edges] un arco tra due utenti $U_1$ e $U_2$, scritto come
  $(U_1 ,U_2)$ rappresenta la relazione di following tra il primo 
  ed il secondo. Di conseguenza, ovviamente, 
  risulta che $U_1 \in $  \followers{}($U_2$) e simmetricamente
   $U_2 \in $  \following{}($U_1$)
   \item[Attachment] ad un arco $e_j \in U\times U$ \`e associato un 
   valore \attach{}($U_1, U_2$) $ \in [0,1]$ che rappresenta l' 
   \textit{attaccamento} di $U_1$ a $U_2$, pi\`u questo valore si 
   avvicina a 0 pi\`u la probabilit\`a che $U_1$ smetta di seguire
   $U_2$ aumenta e viceversa. Quando il nodo viene creato il valore
   di \attach{} \`e relativamente alto ($\sim 0.8$) in quanto ci si 
   aspetta che un utente non smetta di seguire un altro utente
   poco dopo aver iniziato a seguirlo.
\end{description}

\subsection{Probabilit\`a}
\label{subsec:prob}

In tutto il modello sono definite globalmente le 
probabilit\`a di compiere una determinata azione.

\begin{description}
  \item[Tweet] ogni utente U ad ogni tempo T ha la possibilit\`a
  di creare un tweet secondo la seguente probabilit\`a:
    \begin{equation}
    \label{eq:ptwt}
      P_{tweet} = \alpha TZ(T) \dfrac{|followers(U)|}{|Users|}
    \end{equation}
  Questo perch\`e ci si aspetta che un utente popolare sia pi\`u 
  attivo di uno sconosciuto, per mantenere il suo livello di 
  popolarit\`a.
  
  \item[Retweet] similmente al precedente un utente U al tempo T
  ha una possibilit\`a di effettuare un retweet come:
    \begin{equation}
    \label{eq:pretwt}
        P_{retweet} = \beta TZ(T) \dfrac{|followers(U)|}{|Users|}
    \end{equation}
    Risulta da valutare la possibilit\`a che un utente possa fare 
    pi\`u  retweet rispetto al numero dei tweet, in quanto essa \`e
    un'operazione decisamente meno impegnativa e perci\`o,
    intuitivamente,
    con una pi\`u elevata probabilit\`a di accadere. 
    \NB{Stiamo valutando la possibilit\`a di inserire un 
    condizionamento sull'interesse per l'argomento del tweet}
    
  \item[Tag] ogni tweet ha una possibilit\`a di contenere un tag 
  diretto ad un altro utente V con probabilit\`a
  \begin{equation}
  \label{eq:ptag}
    P_{tag} = \gamma \dfrac{follower(U) followers(V)}{{|Users|^2}}
  \end{equation} 
  Ovvero ci si aspetta che il \dtag{} sia proporzionale alla 
  popolarit\`a di entrambi gli utenti coinvolti.
\end{description}

\subsection{Attivit\`a sociali}
\label{subsec:social}

Per prima cosa definiamo gli oggetti che riguardano le attivit\`a 
sociali:

\begin{description}
  \item[Tweet] Il tweet dell'utente U al tempo T \`e definito come
  \twt{}(U,T) = ($j$, \lika{}, \disla{}, \dtag
  %, U, T
  ) dove:
  \begin{itemize}
    \item $j$ \`e il \topic{} su cui il tweet verte
    \item \lika{} $\in [0,1]$ indica la probabilit\`a di quanto
    il tweet possa piacere agli utenti a cui interessa il 
    \topic{} $j$. 
    \item \disla{} rappresenta la probabilit\`a di quanto il tweet 
    possa non piacere a chi non \`e interessato all'argomento
    \item \dtag{} indica l'utente V taggato nel tweet. Tale valore 
    pu\`o anche essere nullo.
%    \item U \`e l'utente che ha effettuato il tweet, e T il tempo in
%    cui \`e stato creato.
%    Necessari per l'implementazione.
  \end{itemize}
  
  \item[Retweet] Il retweet effettuato dall'utente U al tempo T
  del tweet di V al tempo $\overline{T}$, definito come:
  \retwt{}(U, T) = \twt{}(V, $\overline{T}$)
  
  \item[Dtag] Il tag diretto di un utente U ad un altro utente V al 
  tempo T \`e definito come \dtag{}(U, V, T) = tweet(U, T) e 
  rappresenta il caso in cui l' utente U ha ``taggato'' V con un 
  \textit{@U}. Questo permette all'utente V di vedere un tweet 
  di U, anche nel caso in cui non lo seguisse.
\end{description}

In base agli oggetti definiti in precedenza possiamo definire le 
azioni di:

\begin{description}

  \item[Post] Nel momento in cui l'utente U \`e abilitato alla 
  creazione di un tweet allora viene generato casualmente un 
  \topic{} $j$ su cui verter\`a il tweet in modo tale che un topic 
  di interesse per U sia selezionato con maggiore probabilit\`a.
  Una volta scelto il topic $j$:
  \begin{itemize}
    \item se $j$ \`e di interesse per U (U.\PI{}($j$) $\geq 0.5$)
     allora il \twt{} risultante
    avr\`a una \lika{} elevata mentre la \disla{} sar\`a casuale.
    \item viceversa il \twt{} generato avr\`a una \disla{} alta ed
    una \lika{} casuale.
  \end{itemize}
  Una volta definito il tweet viene generata la probabilit\`a di 
  avere un \dtag{} ad un altro utente V in base alla eq. 
  \ref{eq:ptag}. Se tale tag viene generato allora il \twt{} avr\`a
  un tag all'utente V, altrimenti il post non avr\`a alcun \dtag{}.
  
  \item[Repost] Per ogni utente U viene scelta casualmente una lista 
  di $k$ \twt{} $\overline{w} \in$ \FOV{}(U)$^k$, successivamente
  per ogni $\overline{w}_i \in \overline{w}$ viene valutata la 
  possibilit\`a di retweet di $\overline{w}_i$ secondo l' equazione
  \ref{eq:pretwt}, in caso favorevole viene prodotto \retwt{}(U, T) 
  che sar\`a una lista di retweet effettuati da U al tempo T.
  \item[Unfollow] Un utente U pu\`o decidere di smettere di seguire 
  un utente V, grazie alla variabile \attach{} che li lega, e 
  all'ultimo \twt{} $W$ di V con probabilit\`a 
  \begin{equation*}
    P_{unfollow}(U, V) = P(W[dislikability] \mid attachment(U, V))
  \end{equation*}
  Nel caso in cui U decida di continuare a seguire V allora 
  \attach{}(U,V) = $P_{unfollow}(U, V)$ aggiornando dunque la
  probabilit\`a di unfollow a quella appena calcolata.
  Risulta dunque che la probabilit\`a di unfollow dipende solo dallo 
  stato precedente di \attach{}.
    
  \item[Continue Follow] Un utente U pu\`o avere il desiderio di 
  continuare a seguire un altro utente V, se V ha pubblicato un tweet
  riguardante un topic di suo gradimento. Tuttavia tale post ha una
  sua \lika{} che potrebbe pregiudicare, sia in positivo che in 
  negativo, l'\attach{}(U,V) che perci\`o a seguito del nuovo \twt{} 
  W di V risulta che
  \attach{}(U,V) = $P(W[likability] \mid attachment(U,V))$
  
  \item[Nota:] Per le due azioni precedenti di \textbf{Unfollow} e
  \textbf{Continue Follow} bisogna considerare il caso in cui
  la valutazione venga effettuata rispetto ad un retweet e non ad
  un tweet. In questo caso viene comunque effettuata la valutazione
  descritta in precedenza, eventualmente con un influenza minore. 
  
  \item[Follow] Esistono diversi tipi di possibili modalit\`a di 
  following:
    \begin{description}
      \item[By Retweet (BR)] By retweet 
      Nel caso in cui nel FOV compaia un retweet di un utente non 
      seguito, si pu\`o decidere di seguirlo attraverso la seguente 
      valutazione: si considerano gli ultimi $n$ \twt{} e \retwt{} 
      dell'utente target e da questi si inferiscono i suoi interessi.
      Si calcola quindi l'omofilia e, qualora sia minore di una 
      certa soglia, si comincia a seguirlo.  
      \item[Outside Factor (OF)] avviene quando un utente U comincia
      a seguire un nodo V per fattori esterni al Social Network, 
      quali ad esempio nuova amicizia nella vita reale, nuovo
      follow su altri mezzi di comunicazioni online, ecc.
      \NB{Probabilit\`a da definire}
      
      \item[Active Search (AS)] un utente U ricerca un qualunque 
      altro utente V tramite la rete sociale anche se non ha 
      collegamenti con esso. Tale ricerca ha senso secondo il
      modello di omofilia oppure in modo totalmente casuale.
      \NB{Ancora da definire}
            
    \end{description}
\end{description}

\subsection{Step Simulazione}
\label{subsec:step}

Usando il modello definito in precedenza possiamo ora descrivere gli 
step che avvengono nella simulazione al tempo T.

\begin{description}
  \item[Tweet Step (TS)] Per ogni utente U viene generato una 
  probabilit\`a casuale $P_t(U,T)$ che rappresenta la sua 
  inclinazione di produrre un \twt{} al tempo T. Se $P_t(U,T)$
  risulta minore alla probabilit\`a $P_{tweet}$ [eq. \ref{eq:ptwt}] allora
  viene generato il \twt{}(U, T) come descritto in \textbf{Post}.
  
  \item[Retweet Step (RS)] Similmente alla fase precedente,
  per ogni utente U viene generato una 
  probabilit\`a casuale $P_r(U,T)$ che rappresenta la sua 
  inclinazione di produrre un \retwt{} al tempo T. Se $P_r(U,T)$
  risulta minore alla probabilit\`a $P_{retweet}$ [eq. \ref{eq:pretwt}] 
  allora viene generato il \retwt{}(U, T) come descritto in 
  \textbf{Repost}.
  
  \item[Evaluation Step (ES)] Dopo che tutti i \twt{} e i \retwt{}
  sono stati creati al tempo T, per ogni utente U viene generato
  il \FOV{}(U, T) e per ogni $f_i \in$ \FOV{}(U, T) viene valutato 
  l' \attach{}(U, V), dove V \`e l'autore di $f_i$. In base alle 
  probabilit\`a e azioni precedentemente descritte, l'utente U decide
  se smettere di seguire V (secondo \textbf{Unfollow}) o di
  continuare a seguire a seguire, modificando l'\attach{} come
  descritto in \textbf{Continue Follow}.
    Nel caso in cui U stia valutando un retweet di un utente W che non 
  segue, allora valuta la probabilit\`a di seguirlo in base al 
  \textbf{Following BR}.
  Infine viene valutata la possibilit\`a di creare un nuovo follow 
  per \textbf{OF} e \textbf{AS}.
\end{description}

Le azioni descritte in precedenza devono necessariamente essere 
eseguite sequenzialmente una dopo l'altra, tuttavia i singoli step 
possono essere facilmente parallelizzati, senza problemi di 
concorrenza, in quanto il post di un tweet \`e indipendente per ogni
utente; lo stesso vale per il retweet e infine l'aggiornamento dell'
\attach{} \`e  dipendente dai tweet e i retweet effettuati 
\textbf{TS} e \textbf{RS}, ma i singoli aggiornamenti sono 
indipendenti tra gli utenti. Per questo motivo l'idea \`e quella di
fare in modo che le azioni contenute in un step di simulazione 
vengano effettuate con pi\`u thread, ma comunque rispettando l'ordine
descritto.
